%% config
\def\home{../../styles}

%% documentclass
\input{\home/documentclass_normal_oneside}

%% generell-styling
\input{\home/style_proggen}

%% meta-tags for pdf
\newcommand{\pdfauthor}{Matthias Günther}
\newcommand{\pdftitle}{Angular}
\newcommand{\pdfsubject}{My notes about Angular}
\newcommand{\pdfkeywords}{perl}
\newcommand{\motto}{Angular to the rescue}
\newcommand{\tutor}{}
\newcommand{\disclaimer}{(Die Autoren übernehmen keine Garantie und Haftung
für die Korrektheit des Skriptes. Das Skript ist unter den Namen von Matthias
Günther veröffentlich.)}
\newcommand{\publisher}{Der Helex-Matze Verlag $\sum\limits_{i=1}^{n}i$}
\newcommand{\pdfemail}{matthias@wikimatze.de}
\newcommand{\correctiontext}{Kommentare/Korrekturen an}
\newcommand{\homepagetext}{Homepage}
\newcommand{\homepage}{wikimatze.de}
\newcommand{\coverdisclaimer}{Copyright Skript-Covers}
\newcommand{\covercopyright}{\textsc{Ubisoft} (\url{ubi.com})}
\newcommand{\documentwordstext}{Wörter:}
\newcommand{\documentlinestext}{Zeilen:}
\newcommand{\documentwords}{tbd}
\newcommand{\documentlines}{tbd}



%% fancy-header
\input{\home/style_header_oneside}

%% setting the infos for the pdf
\input{\home/info_hypersetup}

%% environments
\input{\home/environments_normal}
\input{\home/environments_mathe}

%% cover
\input{\home/style_cover}


\usepackage[strings]{underscore} % don't escape underscores

% label for programming-language
\input{\home/style_ruby}
\begin{document}
\input{\home/style_starting_document_without_cover}

\section{Angular}
\begin{itemize}
  \item Ist ein MV* für Single-Page Apps
  \item \textbf{Two-Way binding}: d.h. wenn sich das Model ändert, dann verändert sich auch die Seite und umgedreht. Dadurch muss man nicht auf spezifische Elemente achten.
  \item \textbf{dirty checking}: musst keine speziellen Strukturen und getter und setter Methoden schreiben, um Daten zu schreiben und zu lesen
  \item Managed Routing von Views in andere, updated URL im Browser
  \item Erweitert HTML um neue Elemente
\end{itemize}

\begin{verbatim}
<!doctype html>
<html ng-app="app">
    <head>
        <script src="https://ajax.googleapis.com/ajax/libs/angularjs/1.5.0/angular.min.js"></script>
    </head>
  <body>
    <h1 ng-controller="HelloWorldCtrl">{{ helloMessage }}</h1>
    <script type="text/javascript">
        angular.module('app', []).controller('HelloWorldCtrl',

        function($scope) {
            $scope.helloMessage = "Hello, World";
        })
    </script>
  </body>
</html>
\end{verbatim}

\subsection{Controller u. Markup}
\begin{itemize}
  \item kommunizieren mit der View via One-way oder Two-way binding
  \item ist eine Funktion die mein Scope erweitert
  \item nimmt quasi einen leeren scopen und fügt Felder und Funktionen zu, die dann später für die view verwendet wird
  \item ein Controller muss immer in eine App gepackt werden
    \begin{verbatim}
  angular.module('MyHammerLayout', [])
    .controller("index", ["$scope", function($scope)])
    \end{verbatim}
    => definiert einen Controller mit den Namen 'index' und im Array
    werden wieder die Dependencies reingereicht, an dem scope packen wir unsere Funktionen und Variablen aus dem
    Controller
\end{itemize}


\subsection{Services}
\begin{itemize}
  \item kapseln Logik und Zustand der Anwendung
  \item mit denen kommuniziert man mit den Server
\end{itemize}


\subsection{Routing}

\subsection{Direktiven}

\subsection{Testing}

\end{document}
